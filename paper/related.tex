\subsection{DRBD}
\label{sec:drbd}
DRBD\textsuperscript{\textregistered} is a virtual RAID-1 block storage 
device consisting of multiple shared-nothing nodes. The 
DRBD\textsuperscript{\textregistered} interface is implemented as a kernel 
module that can be mounted like a normal block 
device. In this section we focus on replication and synchronization.

\paragraph{Replication}
DRBD\textsuperscript{\textregistered} uses a RAID-1 method of replication, 
meaning there are two nodes in a basic DRBD\textsuperscript{\textregistered} 
cluster --- one \emph{primary} and one \emph{secondary} --- that each
contain typical Linux kernel components\cite{drbd, drbd_manual}. The 
primary node 
services all reads and writes, and the secondary node fully mirrors the 
primary node by propagation of writes across the network. If the primary 
node fails, then service migrates to the secondary node. There are two 
main modes for replication --- fully synchronous and asynchronous. 
The former means that the primary node reports a completed write only after 
the write has been committed to both nodes in the cluster. The latter means 
that each node reports a successful write as soon as the data is written to 
its local disk. 

FreEBS uses a similar technique to the asynchronous method of replication;
we utilize a primary replica that services reads and propagates write 
requests. However, FreEBS waits for a majority of replicas to respond with a
successful write completion. Our system also offers much more flexibility in 
terms of the number of configurable replicas. For instance, 
DRBD\textsuperscript{\textregistered} must use 
stacked DRBD volumes in order to create more than two replicas, which can 
get cumbersome as we add more replicas.

\paragraph{Synchronization} 
DRBD\textsuperscript{\textregistered} offers 
variable-rate and fixed-rate synchronization\cite{drbd_manual}.
For the first method, DRBD\textsuperscript{\textregistered} selects a 
synchronization rate
based on the network bandwidth. For the fixed-rate case, synchronization is
performed periodically at some constant time interval. Synchronization 
can be made more efficient by using checksums to identify blocks that have 
changed since the last synchronization. This eliminates the need to 
synchronize blocks that were overwritten with identical contents. Changes
are tracked using the activity log (AL) which records \emph{hot extents}, 
the blocks that have been modified between synchronization points. The 
activity log uses a quick-sync bitmap to keep track of modified blocks.

FreEBS uses a fixed-rate synchronization method that allows nodes to request 
all writes since an arbitrary version. \emph{Describe how LSVD fetches 
writes here???}

\emph{should we include the DRBD figure here?}

\subsection{Distributed File Systems}
AFS
NFS
Cassandra
Openstack
Eucalyptus

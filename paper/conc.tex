Presently, replicated block stores seem to be dominated by either DRBD-style or commercial NAS solutions. DRBD has the disadvantage that it is somewhat inflexible: it only allows two replicas (without stacking) and lacks dynamically resizing disk images. Further, DRBD puts all of its logic into a Linux kernel driver, making it more difficult to port to different systems. Commercial NAS solutions have the disadvantage of expense and opacity.

FreEBS shows that one can profitably move most of the replication logic into userspace programs. When the kernel driver becomes much simpler, it becomes more portable and reliable. Further, the replica managers themselves are more portable and maintainable. FreEBS is also a strong step in the direction of commoditizing replicated block storage for cloud computing since it enables such technology to be run on commodity hardware.
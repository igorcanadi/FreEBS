FreEBS is a distributed virtual block device that utilizes replication and 
log-based storage to provide reliability and durability. The major 
goal of FreEBS is to create a free and open version of Amazon Elastic Block 
Store (EBS)\cite{amazonEBS}, a virtual mountable block device used by EC2 
instances. Many implementation details about EBS have not been publicized,
but many speculate that EBS is based on 
DRBD\textsuperscript{\textregistered}\cite{drbd}. 

There are two main features that are publicly known about Amazon EBS --- 
replication and snapshots. EBS mirrors volume data across different 
servers within the same Availability Zone. This provides some increased 
reliability over a single disk, but EBS further adds durability using 
snapshots. Snapshots are 
customizable incremental backups stored on Amazon S3. Each snapshot only 
saves the blocks that have been modified since the last snapshot. For each
snapshot there is a table of contents that maps each EBS volume block to 
the valid corresponding block in S3. When a snapshot is deleted, only 
blocks that are not pointed to by any other snapshot are removed from S3. 
Because snapshots only keep track of modified blocks, they take less 
time to perform than a full-volume backup, and require less space to store. 
The frequency at which they are performed determines the durability of 
the volume in the case that the EBS volume fails. \cite{amazonEBS} 

Due to the lack of concrete, detailed information about Amazon EBS, we 
draw many of our ideas about how EBS is implemented from Distributed Replicated 
Block Device\textsuperscript{\textregistered} (DRBD)\cite{drbd}, an open source 
virtual RAID-1 block device. We discuss DRBD\textsuperscript{\textregistered}
in more detail in Section~\ref{sec:related}.

Given what we know of EBS and DRBD, FreEBS seeks to provide the below:

\begin{description}
    \item[Availability]{The system should continue to operate with a certain
            number of replica failures.}
    \item[Durability]{In the case of a failure, the most recent data should
            still exist on the volume.}
    \item[Reliability]{The system should operate as expected and behave
            deterministically.}
    \item[Snapshots]{Our backing file should support snapshots and incremental backups}
\end{description}

We address these goals by using replica servers that coordinate using a
userspace process on each machine. These replicas propagate writes and 
perform synchronization. They also interact with a log-structured virtual 
disk that provides checkpointing, mechanism for recovery, and the 
framework for snapshotting. 

\XXX{Talk briefly about results here?}

The rest of the paper is organized as follows. In Section~\ref{sec:related}
we discuss related work, and in Section~\ref{sec:design} we explain the 
overall design of the system. We go into more detail about our system in 
Section~\ref{sec:implementation}. Finally, we evaluate our design and draw
appropriate conclusions in Section~\ref{sec:evaluation} and 
Section~\ref{sec:conc}, respectively.
